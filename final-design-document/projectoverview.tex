\section{Project Overview}

\subsection{Technical Objectives}

\subsubsection{Batterylife}

The device will be able to remain opperational for six hours disconnected to any
given power supply. This will allow the user to change location and find a place
of inspiration before composing a musical piece

\subsubsection{Energy Consumption}

The device will consume no more than 12 watts in a given moment. This will allow
for an extended battery life to longer times and mitigate the environmental
impact of the MIDI controller

\subsubsection{Processing Time}

The time it takes to compute a given function will take no longer than 5
seconds, limiting user frustration upon percieved lag. This will also force an
efficient solution to the Energy consumption problem as the longer the device
uses the main computing unit the more power it will consume.

\subsection{Goals}

\subsubsection{Function of the Project}

\subsubsection{DAW Integrated Keyboard}

Create a MIDI controller with an integrated and complete Digitial Audio
Workstation(DAW) in the system. The DAW will be able to interact with the user
via the keyboard, which acts as the MIDI controller, and can export a completed
MIDI file to a third party computer.

\subsubsection{Quantization}

The DAW will be able to Quantize a user recording to best fit the percieved
tempo of the recording being played. Users will also be able to adjust said
recording to find the perfect locaiton for each not

\subsubsection{Auto Fill}

The device will be able to fill in a melody to help the user create a unique
piece of music that falls into a specific genre selected.

\subsubsection{Auto Correct}

The device will be able to offer suggestions on a given piece to improve it to
better fit the genre selected.

\subsubsection{Education Mode}

The device will be able to help in the teaching of newer users or muscicians to
master the art of composition and playing melodies.

The final product of this project should serve as a music production and music
education tool. The keyboard itself will allow the user to interactively experiment with
writing melodies, and the autofill feature will assist in expanding the user’s capability.
For music production it can kickstart the writing process, allowing a smooth and
constant flow of ideas and creativity, thereby boosting the productivity of the user.
For music education it can, through demonstration, teach stylistic patterns and
structures that the user can mimic in their own writing to create a similar sound.

\subsubsection{Design Criteria}

Aside from achieving base functionality, there are few requirements regarding the
product’s design. Its keys should be aesthetically and structurally reminiscent of a
piano, to make for more intuitive use. Additionally, the visualization of the melody in
the DAW should follow the horizontal bar format seen in most standard MIDI editors.

\subsubsection{Constraints}

The primary constraint of this project will be hardware. Both the AI model and Digital
Audio Workstation will need to be operating without the assistance of external
processing power. Additionally, we hope to make this product affordable. To meet
these needs, we will need to strike a balance between the complexity of the software
and the capabilities of decently priced hardware.

\subsection{Legal, Ethical, and Privacy Issues}

\subsubsection{Legal}

Copyright of AI generated works is a highly debated issue. It is a common legal opinion
that whoever ran the AI would own the generated work, but there is no legal precedent on
this matter. There are others that believe the creator of the AI would own the copyright
of any generated works by default. We want the end-user to own the copyright to their
generated. We may want to include a clause in our license that specifically states that
the end-user will own the copyright to their generated works. This can hopefully clear up
any concerns to the end-user about ownership and would give users confidence that they can
freely use our device without copyright concerns.

The source code of the model and any other libraries we use should be an open source and
permissive license. Licenses such as MIT and Apache would be more ideal than a copyleft
license, such as the GPL. One of the AI libraries we are evaluating, magenta, uses the
Apache 2.0 License, which is great for us. This license will be easy to comply with, as we
simply need to include a copy of the license with our code and state if there are any
modifications to the code.  That is just one example, but it will be important for us to
audit the licenses of all our dependencies before distribution to be sure we comply.
Especially if we end up using NodeJS, then we may end up with several dependencies
indirectly.

Another area of concern in the realm of copyright is the license of the pretrained model.
If we use a pretrained model we need to verify that we have license to use and distribute
it. The pretrained models would likely not be included in the code repository and could be
subjected to a different license. For this project, we will need the ability to distribute
copies of the pretrained model. We may even have multiple pretrained models per genre if
we accomplish our stretch goal of multiple genres/artists. In such a case we will need to
verify the license of all models for each genre. If we cannot find a pretrained models
that fits this requirement, then we may need to train our own model. If the source code of
the model has a permissive license then it should not be an issue for us to train our own
model for distribution, should that be necessary.

\subsubsection{Ethical}

There are also many questions about the ethics of AI generated artwork. This is a highly
opinionated topic. Many feel great concern that not only does AI have the potential to
substitute labor, but also artistic endeavors. Some who have thought that art would be
safe from AI are shocked to see that AI currently can produce interesting and
thought-provoking art. While it will still be a long time for AI to catch up with human
creativity, it is not a misplaced concern. With that said, we believe that this project
does not fall into this category of AI, as it is intended as a tool to assist musicians
rather than to replace artistic talent. The project’s goal is not to replace the musician,
but to help foster their creativity. The MIDI autofill function will only continue an
existing melody, not generate an entirely new one.

\subsubsection{Privacy}

As far as privacy goes, it is a mostly a non-issue for this project, provided our
implementation is correct. The MIDI device will not require any network connectivity and
will not store any user data. Granted, if we do end up using a microcontroller with
network capability, such as a raspberry pi, we should take extra steps to ensure that any
networking capability is disable so that the host machine cannot get compromised. Although
it would be a bit of a stretch, if the host machine can be compromised through the MIDI
device, then that could lead to major privacy violations.

\subsection{Broader Impacts}
\section{Broader Impacts}

One of the most difficult aspects of producing music as a hobby is a lack of
accessible equipment. Often, it requires several pieces of hardware at over \$100
each as well as licensed software to compose a polished track. This expensive
and convoluted process restricts the reach of music production as an art. As
technology expands into the music production world, the art becomes more
accessible to the everyday person, and the MIDI Autofill keyboard will be yet
another step along that path. With recording, editing, and auto-completion all
conveniently packaged into a single portable piece of equipment, novice
producers can begin completing polished tracks without having to seek out,
invest in, and store a stockpile of hardware.

Additionally, the autofill feature can benefit new practitioners and
professionals alike. One of the biggest stressors in songwriting is writer’s
block – it can completely bottleneck the production process and halt all
progress until resolved. For beginners this can be frustrating and discourage
further exploration into the art, and for professionals this can come at the
detriment of deadlines. The melody autofill feature can provide inspiration or
even an immediate solution, expediting the process and allowing artists to be
more productive with their ideas. The goal is to empower independent creators
because the more people can express themselves through art, the richer and more
diverse our culture and society become.
