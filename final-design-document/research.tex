\section{Research}

\subsection{Raspberry Pi}

As mentioned in \textbf{\ref{sec:embedded_controllers} \nameref{sec:embedded_controllers}},
the Raspberry Pi is mini-computer which can also function as an embedded controller. The
model we're interested in is the Raspberry Pi 4B, which is the latest model and has the
best hardware. After an analysis of other solutions, we have decided that the
Raspberry Pi is the best fit for our project. To that end, we have researched how the
Raspberry Pi can be configured for use in our project.

\subsubsection{Operating System Choice}

The first configuration choice that needs to be made is the operating system. A Raspberry
Pi out of the box has no operating system. An image must be flashed to an SD card to make
the Raspberry Pi useful. Raspberry Pi's typically run some form of Linux. The Raspberry Pi
4B model uses the AArch64 architecture. It can run most AArch64 based Linux distros, but
we must limit ourselves to armv7 (32 bit) operating systems, as tensorflow does not
natively support ARM64 as of the time of writing. Tensorflow lite supports ARM64, but
tensorflow-lite has a limited number of supported operations, which can inhibit our
development. There are still many OS choices available, as there are many flavors of Linux
that can run on the armv7 architecture. Some common choices for Raspberry Pi include
Raspberry Pi OS (formerly known as Raspian), Arch Linux ARM, and DietPi. Each OS has its
advantages and drawbacks, and it is important for us to select a distro that is well
suited for our use case. We will need to modify the system heavily for our purposes, but
we need a good baseline for modification.

We need our distro to be a lightweight bootloader for our application. There are many
examples of highly specialized distros designed for one specific task. Examples
include RetroPIE, which is used to run retro video games, or OpenMediaVault, which is used
to turn the device into a networked storage device. This is similar to what we want our
operating system to do. Our use case for the Raspberry Pi is to run one specialized
application that we ourselves develop. It should boot straight into our application with
no other UI from the operating system. The best solution for us is to have a custom distro
just for running our application. Developing a Linux distro from scratch is a difficult
process that is out of the scope of this project. But what we can do is modify an existing
distro to suite our needs. We plan to do this process using a tool called Packer,
described in more detail in section \ref{sec:packer}.

The most common operating system used for Raspberry Pi is Raspberry Pi OS. This is a
specially tailored version of Debian for the Raspberry Pi. This is a common choice for any
general purpose usage the Pi. It can be used like a normal desktop computer when a mouse,
keyboard, and monitor are plugged in. This is a "batteries included" distro that includes
many things we will never use. The bloatware from this distro can hinder performance and
waste SD card space. There is a lightweight alternative known as Raspberry Pi OS Lite.
This version does not include any desktop environment. We think it would be better to
start with an even lighter weight distro, such as DietPi.

DietPi is like Raspberry Pi OS Lite in that it is a lightweight distro. DietPi's claims to
be even lighter than Raspberry Pi Os Lite. It occupies 589 Mb on the SD card, while
Raspberry Pi OS occupies 1424 MB. There are other system optimizations in place to improve
general performance as well. After boot, there are 11 total processes, versus 18 total
processes running on Raspberry Pi OS. This is a good choice for base system. It is easier
for us to start lightweight and add what we need, rather than to start with a bloated
system and strip it down. DietPi includes an easy way to run chromium in kiosk mode using
the \url{dietpi-autostart} utility, which we'll need to have configured anyways, as
described in section \ref{sec:research:subsec:os_config}. DietPi also supports a fully
automated installation process, which we intend to take advantage of, as it is tedious to
have to flash the Raspberry Pi and go through the distro's installation process every time
we change the operating system configuration.

From our research, we have decided to use DietPi as a baseline operating system for our
project. The lightweight aspect suits us well, as we only need to boot a single
application. The easy chromium kiosk mode configuration is also a big bonus, as it
relieves us from a configuration burden.

\subsubsection{Operating System Configuration}
\label{sec:research:subsec:os_config}

The operating system will need to be configured for our use-case. We need to get the
operating system to boot a single GUI application, our DAW, without displaying any other
UI from the desktop environment (DE). There may be more applications separate from the DAW
that will run in the background which also must  immediately after boot. The
operating system will need to be configured to not permit any networking out of security
and privacy concerns.

To only display a single application, we need to configure the graphics system in our
distro. Graphics in Linux are done through the X Window System. In modern apps, X is
mostly agnostic to the UI, and is used for providing UI toolkits a method for accessing
bitmaps to windows. X can be configured in a variety of ways. The way we intend to
configure X is to host a single fullscreen application for our DAW. This can be done
through the \url{.xinitrc} file. This file specifies which commands to run whenever the X
server starts. By including the command to run our application in there, we will have a
single window at startup.

\begin{lstlisting}[language=bash, label={lst:xinitrc}, caption=Example .xinitrc]
#!/bin/sh

exec chromium --kiosk http://127.0.0.1:8080/
\end{lstlisting}

Because our app is browser based, the startup application will be a web browser. In our
case we want to use the Chromium browser, which is the open source variant of Google
Chrome. Chromium has a feature called Kiosk mode, which puts the browser in fullscreen
with no border or frame \autocite{chromiumKioskMode}. This in combination with the xinitrc
file fulfills our requirement of only having a single fullscreen application with no other
UI. This feature can be enabled through the \url{--kiosk} command line flag
\autocite{chromiumKioskMode}. See \nameref{lst:xinitrc} for an example of an xinitrc that
launches chromium in fullscreen mode. DietPi can also handle this configuration for us,
using the \url{dietpi-autostart} configuration utility.

Networking can disabled through the \url{/etc/network/interfaces} config file. This config
file is used to declare all networking interfaces, and the interfaces can be removed by
simply removing all lines declaring a networking interface in that file. We can remove all
networking interfaces except for the loopback (\url{lo}) and USB Ethernet interfaces.

There is more operating system configuration that must happen as well in order for the
Raspberry Pi 4B to be configured as USB MIDI Gadget. This bit of configuration is quite
extensive and is covered in more detail in section \ref{sec:midi_peripheral}.

\subsubsection{Packer}
\label{sec:packer}

A solution we found for customizing an existing Linux distro is Packer. Packer is a
utility that can take an existing operating system image, change it, and produce a new
image. Modification is done through provisioners, which exist to perform steps, such as
copying files, running shell commands, setting permissions, etc. Packer relies on builders
to carry out the tasks laid out in a configuration file. The builder we will use is called
packer-builder-arm, which builds ARM images and is suitable for a Raspberry Pi.

Packer is configured through simple JSON files. These JSON files contain information about
the original operating system's image, such as the URL it's hosted on, as well as other
metadata about the image. The JSON file also configures provisioners to modify the
operating system. With this provisioners, we can create our own packer json config file to
setup all the tasks described in \nameref{sec:research:subsec:os_config}.

One of the provisioners we will use is the file provisioner, which copies files from the
build tree over to the operating system. For example, the file provisioner can copy the
\url{.xinitrc} file to the home directory, which configures X11 for our application.
We will also use file provisioners to copy over the \url{/boot/} modifications required to
turn the device into a USB gadget, as described in section \ref{sec:midi_peripheral}.

We will use shell provisioners to run all the DietPi installation and configuration steps.
This will include automating the options we specify in the \url{dietpi-config} and
\url{dietpi-software} utilities. We will also use shell provisioners to clone our DAW
application and any other backend software we need to run on the device. The software
should be pulled straight from the git repository if available.

The final output is a reliable \url{.img} file built, that can be flashed to an SD card
using a simple image flashing utility, such as balenaEtcher. This image will already have
the operating system fully configured and with our software preinstalled onto it, so there
will be no need to do any setup after the image is flashed.

We are also looking into having this packer build be integrated into our GitHub repo with
GitHub Actions, which will automatically build a new image on every pull request. This
will allow us to ensure that the OS build is working before merging any new code into our
upstream. This should give us the confidence to work on the code without fear that we will
break the Raspberry Pi image builds every time.

\subsubsection{Performance}

Performance is always a concern when dealing with constrained embedded hardware. The
Raspberry Pi fares better than microcontrollers you would find in a typical MIDI
controller, but is still performance constrained. Our device needs to be able to run a
machine learning model and a UI application, both of which are computationally expensive
tasks.

\paragraph{Magenta.}

Magenta serves as a good baseline for measuring performance of music generating AI.
Magenta ships with two machine learning models: MusicRNN and MusicVAE. We have performance
tested both of these in a demo benchmark program on the Raspberry Pi 4B.

The benchmark program we constructed uses Magenta with JavaScript. We use the Tensorflow
node backend for good CPU performance with Node. The program uses both the MusicRNN and
MusicVAE models, with checkpoints that are hosted are Google's servers. The checkpoints
tested are the \url{basic_rnn}, \url{melody_rnn}, and \url{drum_kit_rnn} checkpoints for
the MusicRNN, and the \url{mel_4bar_small_q2} checkpoint for \url{MusicVAE}. The benchmarks
use a JavaScript benchmarking framework called Benny to run several test cases with
magenta, sampling each one several times and providing statistics such as the min, max,
and mean generation time, as well as the standard deviation.

Our benchmarking program has two benchmarking suites: a generalized one to gauge overall
performance (see \nameref{appendix:magenta_benchmark}) and a linear suite to find out what
factors affect performance the most.

In the generalized suite, the MusicRNN model with the provided pretrained checkpoints has
shown good performance overall when autofilling simple and random melodies, well below the
5 second threshold in general with a few edge cases that exceed 5 seconds. With this
knowledge we know its feasible to run a music generating AI on the Raspberry Pi. What we
need to know is how much we are able to push the music generating AI before we fall below
our threshold. This is where the linear benchmarks come in.

The linear benchmarks allow us to figure out what constraints we have to limit ourselves
to in order to stay below our 5 second threshold. We based these benchmarks around factors
we speculated would affect the performance the most. For each of these benchmark factors,
we progressively test higher amounts and graph the results for easy analysis (see
\autoref{fig:magentaperf}).

\paragraph{Linear benchmark factors}
\begin{itemize}
  \item \textbf{Melody duration.} This is the duration of the input melody in seconds.
  \item \textbf{Number of notes.} Total number of notes in the input melody. This is with
        a constant duration of 16 seconds.
  \item \textbf{Generation steps.} Each generation step represents a quarter note in these
        benchmarks. The duration is constant at 16 seconds.
  \item \textbf{Temperature} This is the temperature of the RNN, which affects the
        randomness of the generated melodies.
\end{itemize}

All benchmarks in this suite use random notes quantized to quarter steps. The random notes
fit the checkpoints supported note rage, which is different between checkpoints. For
example, the \url{basic_rnn} checkpoint supports pitches in the range \url{[48, 84]},
while \url{melody_rnn} supports the full MIDI range of \url{[0, 127]}.

With this data collected, it becomes easy to analyze what constraints we are working
with. Our target is to remain below 5 seconds of generation time, so we need to ensure
that we have a safe margin below where the graphs meet 5 seconds. The generation time
increases linearly with melody duration and total generation steps. The \url{basic_rnn}
and \url{melody_rnn} checkpoints were neck and neck in terms of generation time, while the
\url{drum_kit_rnn} checkpoint was consistently slower. The temperature had no effect on
the generation time. We were initially surprised to see that the total number of notes had
no effect on generation time, which remained constant when increasing from 100 to 2000
input notes, but this makes sense as the input tensor size does not change with the number
of notes.

The input duration is the biggest factor affecting generation time. The number of
generation steps are the next biggest factor. With this data, we know that we will most
likely need to truncate the input melody's duration to less than a 40 seconds of music.
The generation steps need to be accounted for when determining the duration of music to
accept as input for our AI. As the generation steps increase we need make up for that
amount of extra time by decreasing the input duration even further. We added in a
benchmark for increasing duration and generation steps simultaneously to see if this would
lead to exponential behavior, but found that it remained linear. So we can simply subtract
the estimated added time from the input melody duration.

With these benchmark results, we are confident that the Raspberry Pi 4B is capable of
handling the computational load that this project requires, without the need for any
external processing. We also have a good benchmarking solution that can be adapted for any
checkpoints or models that we will further into the project's development.

\begin{figure}
  \centerline{ \includegraphics[width=.95\linewidth]{image/perf.pdf} }
  \caption{Magenta Generation Time on the Raspberry Pi 4B}
  \label{fig:magentaperf}
\end{figure}

\paragraph{UI.} It is important to verify that a Raspberry Pi can even run a browser based
DAW with acceptable performance before spending development time creating the web app.
Since we have not developed the DAW at this point, we have tested an existing web based
MIDI editor in the Raspberry Pi.

We used an online MIDI editor called signal (\url{https://signal.vercel.app/}) to get a
subjective feel for the performance. The test was ran in the chromium web browser on
Raspberry Pi OS. The performance was acceptable, with some actions feeling slow.
Dragging around notes had a small but noticeable latency between the mouse cursor's
movement and the notes position. There was also a small but noticeable latency between
double clicking to insert a note and the note showing up on screen. The playhead updated
smoothly when playing the MIDI track, which is the most important aspect, as laggy
playback is awful for the user experience.

It should be noted that this app was not explicitly designed to run on constrained
hardware. Web apps can run great on Raspberry Pi's when optimized to do so. Based on our
findings here we are confident that we can develop a browser based app that has
exceptional performance on the Raspberry Pi.

\subsection{MIDI Peripheral}
\label{sec:midi_peripheral}

The Pi needs to function as a MIDI peripheral, i.e. plug the device into a computer and it
functions as a device that sends MIDI commands to the DAW. This connection will happen
through a USB-C to USB-A cable from the Raspberry Pi to the computer. The Raspberry Pi is
typically powered from the USB-C port form a power supply connected to a power outlet, but
in our case the Pi will receive it's power from the host PC over USB-C instead. Once
plugged in, the Pi will act as a normal MIDI peripheral, which sends MIDI events to the
PC, which can be picked up by a fully fledged DAW.

The MIDI communication happens over USB in our case, which means we need to work with the
operating system to send these messages. We are using Linux for our device's operating
system, so USB communication needs to happen through the Linux kernel. This is known as a
USB Gadget in the Linux kernel \autocite{usbGadgetDocumentation}. The Raspberry Pi 4b can
act as a Linux USB Gadget through a USB-C to USB-A connection to the computer, with USB-C
being on the Raspberry Pi and USB-A being on the host computer
\autocite{raspberryPiGadgetSetup}. This will not work out of the box, as the Linux kernel
needs to be made aware that we intend to use the device as a USB gadget. This
configuration happens through files in the \url{/boot/} directory
(see listings \ref{lst:bootconfig} and
\ref{lst:bootcmdline}) \autocite{raspberryPiGadgetSetup}. The Pi will act as a USB
gadget after Linux boot configuration settings are set and it is plugged up to the host
machine.

\begin{minipage}{\linewidth}

  \begin{lstlisting}[language=bash,
  label={lst:bootconfig},
  caption=Lines added to /boot/config.txt \autocite{raspberryPiGadgetSetup, raspberryPiHDMIFix}]
dtoverlay=dwc2
hdmi_force_hotplug=1
hdmi_group=2
hdmi_mode=87
hdmi_cvt=800 480 60 6 0 0 0
hdmi_drive=1
  \end{lstlisting}

  \begin{lstlisting}[language=bash, label={lst:bootcmdline}, caption=DietPi /boot/cmdline.txt modified to allow a USB Gadget \autocite{raspberryPiGadgetSetup}, breaklines=true]
console=ttyS0,115200 console=tty1 root=PARTUUID=8f4dbd00-02 rootfstype=ext4 elevator=deadline fsck.repair=yes rootwait quiet net.ifnames=0 modules-load=dwc2,g_ether
  \end{lstlisting}

  \begin{lstlisting}[label={lst:usb_gadget}, caption=Bash procedure to setup a MIDI Gadget\, modified for our device \autocite{raspberryPiGadgetSetup}, breaklines=true]
cd /sys/kernel/config/usb_gadget/
mkdir -p midi_over_usb
cd midi_over_usb
echo 0x17e8  > idVendor
echo 0xb09c > idProduct
echo 0x0100 > bcdDevice
echo 0x0200 > bcdUSB
mkdir -p strings/0x409
echo "fedcba9876543210" > strings/0x409/serialnumber
echo "MIDI Autofill Group" > strings/0x409/manufacturer
echo "MIDI Autofill Device" > strings/0x409/product
  \end{lstlisting}

\end{minipage}

After the boot configuration options are set, the device will be a USB gadget, but there
is more configuration needed for it to act as a MIDI USB gadget. The Linux kernel already
provides an interface for USB gadgets to act as MIDI devices, through the
\url{usb_f_midi.ko} kernel module \autocite{usbGadgetDocumentation}. We need to set our
Raspberry Pi's USB gadget settings to identify itself as a MIDI peripheral. We have
followed an online guide to setup this MIDI gadget, which involves providing information
to the kernel about our device through the \url{/sys/kernel/config/usb_gadget/} folder
\autocite{raspberryPiGadgetSetup}.

From here we can use the \url{aplaymidi} command to test our MIDI device, which sends a
MIDI file to the host computer \autocite{gadgetTesting}. If the gadget is working
properly, then the MIDI sequences should be visible in any application that listens for
MIDI note events, such as a DAW.

USB Gadgets appear as USB ethernet devices and can be read and written to like any other
networking interface. USB MIDI gadgets similarly communicate through standard networking
interfaces under Linux, meaning that our backend program for sending MIDI events to the
host computer will be sending messages through networking interfaces. However, we will
most likely use a library to abstract away this networking aspect.

USB gadgets work, but we ran into a problem with our setup where our HDMI connection to
the monitor was not working anymore. This connection is critical, as we need to be able to
display our DAW to the built in connection using HDMI, so we had to find a workaround for
this issue. We found out that there are \url{/boot/config.txt} options that can added to
force an HDMI connection, using the \url{hdmi_force_hotplug} option
\autocite{raspberryPiHDMIFix}. This configuration as well as some other HDMI settings can
be seen in listing \ref{lst:bootconfig}.

We have also ran into some low voltage warnings when running the Raspberry Pi 4B as a USB
gadget. These warnings are found in system logs and are viewed with \url{journalctl}. The
Raspberry Pi 4b requires 5.1 volts and 3 amps to operate effectively, and will send the
undervolt warning whenever it runs below 4.63 volts
\autocite{raspberryPiAmps}. We notice that we see these warnings early in the boot, but it
seems to stabilize later. We have not run into any issues with these undervolts as of now.
The Pi does not encounter anymore voltage warnings when running a CPU stress test, but it
could lead to stability issues later.  We are still researching  solutions to this
problem.
