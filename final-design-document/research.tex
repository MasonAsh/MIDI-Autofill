\section{Research}

\subsection{AI}

\blindtext

\subsubsection{PixelCNN}

\blindtext

\subsubsection{Magenta}

\blindtext

\subsection{Embedded Controller Comparisons}
\label{sec:embedded_controllers}

\blindtext

\subsection{DAW Frontend Technologies}

\blindtext

\subsection{Raspberry Pi Implementation Details}

As mentioned in \nameref{sec:embedded_controllers}, the Raspberry Pi is mini-computer
which can also function as an embedded controller. The model we're interested in is the
the Raspberry Pi 4B. This Raspberry Pi model uses the ARM64 architecture. It can run
numerous ARM64 based Linux distros. The advantage here is the software and tooling that is
available for us to run in a Linux environment. After an analysis of other solutions, we
have decided that the Raspberry Pi is the best fit for our project. To that end, we have
researched how the Raspberry Pi can be configured for use in our project.

\subsubsection{Operating System Choice}

The first configuration choice that needs to be made is the operating system. A Raspberry
Pi out of the box has no operating system. An image must be flashed to an SD card to make
the Raspberry Pi useful. Raspberry Pi's typically run some form of Linux. There are a
number of choices available, as there are many flavors of Linux that can run on the ARM64
architecture. Some common choices for Raspberry Pi include Raspberry Pi OS (formerly known
as Raspian), Arch Linux ARM, and DietPi. Each OS has its advantages and drawbacks, and it
is important for us to select a distro that is well suited for our use case.

The most common operating system used for Raspberry Pi is Raspberry Pi OS. This is a
specially tailored version of Debian for the Raspberry Pi. This is a common choice for any
general purpose usage the Pi. It can be used like a normal desktop computer when a mouse,
keyboard, and monitor are plugged in. This is a "batteries included" distro that includes
many things we will never use. The bloatware from this distro can hinder performance and
waste SD card space. There is a lightweight alternative known as Raspberry Pi OS Lite.
This version does not include any desktop environment.

DietPi is similar to Raspberry Pi OS Lite in that it aims to be a lightweight distro.
DietPi's website claims to be even lighter than Raspberry Pi Os Lite. It occupies 589 Mb
on the SD card, while Raspberry Pi OS occupies 1424 MB. There are a number of other system
optimizations in place to improve general performance. After boot, there are only 11 total
processes, versus 18 total processes running on Raspberry Pi OS.

There are also specialized distros that are only meant to do one specific thing. Examples
include RetroPIE, which is used to run retro video games, or OpenMediaVault, which is used
to turn the device into a networked storage device. This is similar to what we want our
operating system to do. Our use case for the Raspberry Pi is to run one specialized
application that we ourselves develop. It should boot straight into our application with
no other UI from the operating system. The best solution for us is to have a custom distro
just for running our application. Developing a Linux distro from scratch is a difficult
process that is out of the scope of this project. But what we can do is modify an existing
distro to suite our needs.

\subsubsection{Operating System Configuration}
\label{sec:research:subsec:os_config}

The operating system will need to be configured for our use-case. We need to get the
operating system to boot a single GUI application, our DAW, without displaying any other
UI from the desktop environment (DE). There may be more applications separate from the DAW
that will run in the background which also must be started immediately after boot. The
operating system will need to be configured to not permit any networking out of security
and privacy concerns.

To only display a single application, we need to configure the graphics system in our
distro. Graphics in Linux are done through the X Window System. In modern apps, X is
mostly agnostic to the UI, and is used for providing UI toolkits a method for accessing
bitmaps to windows. X can be configured in a variety of ways. The way we intend to
configure X is to host a single fullscreen application for our DAW. This can be done
through the \url{.xinitrc} file. This file specifies which commands to run whenever the X
server is started. By simply including the command to run our application in there, we
will have an single window at startup.

Networking can disabled through the \url{/etc/network/interfaces} config file. This config
file is used to declare all networking interfaces, and the interfaces can be removed by
simply removing all lines declaring a networking interface in that file.

\subsubsection{Packer}

A solution we found for customizing an existing Linux distro is Packer. Packer is a
utility that can take an existing operating system image, modify it, and produce a new
image. Modification is done through provisioners, which exist to perform steps, such as
copying files, running shell commands, setting permissions, etc. Packer relies on builders
to carry out the tasks laid out in a configuration file. The builder we will use is called
packer-builder-arm, which builds ARM images and is suitable for a Raspberry PI.

We can use our own Packer config file to setup all the tasks described in
\nameref{sec:research:subsec:os_config}. For example, we can use the file provisioner to
copy the \url{.xinitrc} file to the home directory, which is used to configure X11 for our
application.

\subsection{Keyboard Design}
