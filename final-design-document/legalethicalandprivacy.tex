\subsection{Legal, Ethical, and Privacy Issues}

\subsubsection{Legal}

Copyright of AI generated works is a highly debated issue. It is a common legal opinion
that whoever ran the AI would own the generated work, but there is no legal precedent on
this matter. There are others that believe the creator of the AI would own the copyright
of any generated works by default. We want the end-user to own the copyright to their
generated works. We may want to include a clause in our license that specifically states
that the end-user will own the copyright to their generated works. This can hopefully
clear up any concerns to the end-user about ownership and would give users confidence that
they can freely use our device without copyright concerns.

The source code of the model and any other libraries we use should be an open source and
permissive license. Licenses such as MIT and Apache would be more ideal than a copyleft
license, such as the GPL. One of the AI libraries we are evaluating, magenta, uses the
Apache 2.0 License, which is great for us. This license will be easy to comply with, as we
simply need to include a copy of the license with our code and state if there are any
modifications to the code.  That is just one example, but it will be important for us to
audit the licenses of all our dependencies before distribution to be sure we comply.
Especially if we end up using NodeJS, then we may end up with several dependencies
indirectly.

Another area of concern in the realm of copyright is the license of the pretrained model.
If we use a pretrained model we need to verify that we have license to use and distribute
it. The pretrained models would likely not be included in the code repository and could be
subjected to a different license. For this project, we will need the ability to distribute
copies of the pretrained model. We may even have multiple pretrained models per genre if
we accomplish our stretch goal of multiple genres/artists. In such a case we will need to
verify the license of all models for each genre. If we cannot find a pretrained models
that fits this requirement, then we may need to train our own model. If the source code of
the model has a permissive license then it should not be an issue for us to train our own
model for distribution, should that be necessary.

\subsubsection{Ethical}

There are also many questions about the ethics of AI generated artwork. This is a highly
opinionated topic. Many feel great concern that not only does AI have the potential to
substitute labor, but also artistic endeavors. Some who have thought that art would be
safe from AI are shocked to see that AI currently can produce interesting and
thought-provoking art. While it will still be a long time for AI to catch up with human
creativity, it is not a misplaced concern. With that said, we believe that this project
does not fall into this category of AI, as it is intended as a tool to assist musicians
rather than to replace artistic talent. The project’s goal is not to replace the musician,
but to help foster their creativity. The MIDI autofill function will only continue an
existing melody, not generate an entirely new one.

\subsubsection{Privacy}

As far as privacy goes, it is a mostly a non-issue for this project, provided our
implementation is correct. The MIDI device will not require any network connectivity and
will not store any user data. Granted, if we do end up using a microcontroller with
network capability, such as a raspberry pi, we should take extra steps to ensure that any
networking capability is disable so that the host machine cannot get compromised. Although
it would be a bit of a stretch, if the host machine can be compromised through the MIDI
device, then that could lead to major privacy violations.